\documentclass[]{article}

\usepackage{amsmath}  % AMS math package
\usepackage{amssymb}  % AMS symbol package
\usepackage{bm}       % bold math
\usepackage{graphicx} % Include figure files
\usepackage{dcolumn}  % Align table columns on decimal point
\usepackage{multirow} % Multirow/column tables
\usepackage{hyperref} % Hyperlinks

\begin{document}

\title{Manuscript Title:\\with Forced Linebreak}% Force line breaks with \\
\author{Ann Author}
\date{\today}% It is always \today, today, but you can specify other dates manually 
\maketitle

\begin{abstract}
An article usually includes an abstract, a concise summary of the work
covered at length in the main body of the article. 
\end{abstract}


\section{First-level heading} %Title for the section
\label{sec:level1} %Label for the section, to be used for referencing in other parts of the document

This sample document demonstrates some common uses of \LaTeX\ in documents you'll write for ICCP. Further information can be found online at the \href{http://en.wikibooks.org/wiki/LaTeX}{\LaTeX\ wikibook} or in the distribution documentation. 

When we refer to commands in this example file, they always have their required formal arguments in normal \TeX{} format. In this format, \verb+#1+, \verb+#2+, etc. stand for required author-supplied arguments to commands. For example, in \verb+\section{#1}+ the \verb+#1+ stands for the title text of the author's section heading, and in \verb+\title{#1}+ the \verb+#1+ stands for the title text of the paper.

Line breaks in section headings at all levels can be introduced using \textbackslash\textbackslash. A blank input line tells \TeX\ that the paragraph has ended. 

\subsection{\label{sec:level2} Second-level heading: Formatting}

This file may be formatted in either the \texttt{preprint} or \texttt{reprint} style. \texttt{reprint} format mimics final journal output. The paper size may be specified with the option \texttt{letter}. These options are inserted in the square brackets inside the \texttt{\textbackslash documentclass[]\{article\}} command.

\section{Math and Equations}
Inline math may be typeset using the \verb+$+ delimiters. Bold math symbols may be achieved using the \verb+bm+ package and the \verb+\bm{#1}+ command it supplies. For instance, a bold $\alpha$ can be typeset as \verb+$\bm{\alpha}$+ giving $\bm{\alpha}$. Fraktur and Blackboard (or open face or double struck) characters should be typeset using the \verb+\mathfrak{#1}+ and \verb+\mathbb{#1}+ commands respectively. Both are supplied by the \texttt{amssymb} package. For
example, \verb+$\mathbb{R}$+ gives $\mathbb{R}$ and \verb+$\mathfrak{G}$+ gives $\mathfrak{G}$

In \LaTeX\ there are many different ways to display equations, and a few preferred ways are noted below. Displayed math will center by default. Use the class option \verb+fleqn+ to flush equations left.

Below we have numbered single-line equations; this is generally the most common type of equation in \LaTeX: 
\begin{equation} 
\label{eq:one} %Label for the equation, to be used for referencing in other parts of the document
  i \hbar \frac{\partial \Psi}{\partial t} = -\frac{\hbar^2}{2m}\nabla^2 \psi + U(\mathbf{x}) \psi
\end{equation}

When the \verb+\label{#1}+ command is used [cf. input for Eq.~(\ref{eq:one})], the equation can be referred to in text without knowing the equation number that \TeX\ will assign to it. Just use \verb+\ref{#1}+, where \verb+#1+ is the same name that was used in the \verb+\label{#1}+ command.

Unnumbered single-line equations can be typeset using the \verb+equation*+ environment:
\begin{equation*}
  \hat{f}(\xi) = \int_{-\infty}^\infty f(x) e^{-2\pi i x \xi} \, \mathrm{d}x
\end{equation*}

\subsection{Multiline equations}

You'll generally want to use the \verb+align+ environment for multiline equations. Use the \verb+\nonumber+ command at the end of each line (again denoted with \textbackslash\textbackslash) to avoid assigning a number, or \verb+align*+ to disable numbering entirely:
\begin{align}
  \sin(2\theta) &= 2 \sin(\theta) \cos(\theta) \nonumber \\
                &= \frac{2\tan(\theta)}{1 + \tan^2(\theta)}
\end{align}
Note how alignment occurs at the (unprinted) \verb+&+ character.

Do not use \verb+\label{#1}+ on a line of a multiline equation if \verb+\nonumber+ is also used on that line. Incorrect cross-referencing will result. Notice the use \verb+\text{#1}+ for using a Roman font within a math environment.

\section{Cross-referencing}
\LaTeX\ will automatically number such things as sections, footnotes, equations, figure captions, and table captions for you. In order to reference them in text, use the \verb+\label{#1}+ and \verb+\ref{#1}+ commands\footnote{Check out the \texttt{cleveref} (one r) package, too!}. To reference a particular page, use the \verb+\pageref{#1}+ command.

The \verb+\label{#1}+ should appear within the section heading (or immediately after), within the footnote text, within the equation, or within the table or figure caption. The \verb+\ref{#1}+ command is used in text at the point where the reference is to be displayed.  Some examples: Section~\ref{sec:level1} on page~\pageref{sec:level1}, Table~\ref{tab:table1}, and Fig.~\ref{fig:epsart}.
\begin{figure}[b]
  \centering
  \includegraphics{figures/fig_1}% Imports a figure - does not automatically scale
  \caption{\label{fig:epsart} A figure caption. The figure captions are automatically numbered.}
\end{figure}

\section{Floats: Figures, Tables, etc.}
Figures (images) and tables are usually allowed to ``float'', which means that their placement is determined by \LaTeX, while the document is being typeset. 

Use the \texttt{figure} environment for a figure, the \texttt{table} environment for a table. In each case, use the \verb+\caption+ command within to give the text of the figure or table caption along with the \verb+\label+ command to provide a key for referring to this figure or table. Insert an image using either the \texttt{graphics} or \texttt{graphix} packages, which define the \verb+\includegraphics{#1}+ command. (The two packages differ in respect of the optional arguments used to specify the orientation, scaling, and translation of the image.) To create an alignment, use the \texttt{tabular} environment. 

\begin{table}
  \centering
  \begin{tabular}{ |l|l|l| }
  \hline
  \multicolumn{3}{ |c| }{Team sheet} \\
  \hline
  Goalkeeper & GK & Paul Robinson \\ \hline
  \multirow{4}{*}{Defenders} & LB & Lucus Radebe \\
   & DC & Michael Duburry \\
   & DC & Dominic Matteo \\
   & RB & Didier Domi \\ \hline
  \multirow{3}{*}{Midfielders} & MC & David Batty \\
   & MC & Eirik Bakke \\
   & MC & Jody Morris \\ \hline
  Forward & FW & Jamie McMaster \\ \hline
  \multirow{2}{*}{Strikers} & ST & Alan Smith \\
   & ST & Mark Viduka \\
  \hline
  \end{tabular}
  \caption{\label{tab:table1}A table, demonstrating the use of the \texttt{multirow} package for spanning rows and columns.}
\end{table}

The best place for the \texttt{figure} or \texttt{table} environment is immediately following its first reference in text; this sample document illustrates this practice for Fig.~\ref{fig:epsart}, which shows a figure that is small enough to fit in a single column. In exceptional cases, you will need to move the float earlier in the document: \LaTeX's float placement algorithms need to know about a full-page-width float sooner.

The content of a table is typically a \texttt{tabular} environment, giving rows of type in aligned columns. Column entries separated by \verb+&+'s, and 
each row ends with \textbackslash\textbackslash. The required argument for the \texttt{tabular} environment specifies how data are aligned in the columns. 
For instance, entries may be centered, left-justified, right-justified, aligned on a decimal point. 

Extra column-spacing may be be specified as well, although \LaTeX sets this spacing so that the columns fill the width of the table. Horizontal rules are typeset using the \verb+\hline+ command. Rows with that columns span multiple columns can be typeset using the \verb+\multicolumn{#1}{#2}{#3}+ command (for example, see the first row of Table~\ref{tab:table1}).

\appendix

\section{Appendixes}

To start the appendixes, use the \verb+\appendix+ command. This signals that all following section commands refer to appendixes instead of regular sections. Therefore, the \verb+\appendix+ command should be used only once---to setup the section commands to act as appendixes. Thereafter normal section commands are used. The heading for a section can be left empty. For example, 
\begin{verbatim}
\appendix
\section{}
\end{verbatim}
will produce an appendix heading that says ``APPENDIX A'' and
\begin{verbatim}
\appendix
\section{Background}
\end{verbatim}
will produce an appendix heading that says ``APPENDIX A: BACKGROUND'' (note that the colon is set automatically).

If there is only one appendix, then the letter ``A'' should not appear. This is suppressed by using the star version of the appendix command (\verb+\appendix*+ in the place of \verb+\appendix+).

\section{A little more on appendixes}

Observe that this appendix was started by using
\begin{verbatim}
\section{A little more on appendixes}
\end{verbatim}

Note the equation number in an appendix:
\begin{equation}
  E^2=p^2c^2 + m^2c^4.
\end{equation}

\subsection{\label{app:subsec}A subsection in an appendix}

You can use a subsection or subsubsection in an appendix. Note the numbering: we are now in Appendix~\ref{app:subsec}.

Note the equation numbers in this appendix, produced with the subequations environment:
\begin{subequations}
\begin{align}
  E^2 &= m^2c^4 \quad \text{(rest)}, \label{appa} \\
  E^2 &= m^2c^4 + p^2 c^2 \quad \text{(relativistic)}, \label{appb} \\
  E^2 &\approx p^2 c^2 \quad \text{(ultrarelativistic)} \label{appc}
\end{align}
\end{subequations}
They turn out to be Eqs.~(\ref{appa}), (\ref{appb}), and (\ref{appc}).

\end{document}
